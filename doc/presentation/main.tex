%%%%%%%%%%%%%%%%%%%%%%%%%%%%%%%%%%%%%%%%%
% Beamer Presentation
% LaTeX Template
% Version 1.0 (10/11/12)
%
% This template has been downloaded from:
% http://www.LaTeXTemplates.com
%
% License:
% CC BY-NC-SA 3.0 (http://creativecommons.org/licenses/by-nc-sa/3.0/)
%
%%%%%%%%%%%%%%%%%%%%%%%%%%%%%%%%%%%%%%%%%

%----------------------------------------------------------------------------------------
%	PACKAGES AND THEMES
%----------------------------------------------------------------------------------------

\documentclass{beamer}

\mode<presentation> {

% The Beamer class comes with a number of default slide themes
% which change the colors and layouts of slides. Below this is a list
% of all the themes, uncomment each in turn to see what they look like.

%\usetheme{default}
%\usetheme{AnnArbor}
%\usetheme{Antibes}
%\usetheme{Bergen}
%\usetheme{Berkeley}
%\usetheme{Berlin}
%\usetheme{Boadilla}
%\usetheme{CambridgeUS}
%\usetheme{Copenhagen}
%\usetheme{Darmstadt}
%\usetheme{Dresden}
%\usetheme{Frankfurt}
%\usetheme{Goettingen}
%\usetheme{Hannover}
%\usetheme{Ilmenau}
%\usetheme{JuanLesPins}
%\usetheme{Luebeck}
\usetheme{Madrid}
%\usetheme{Malmoe}
%\usetheme{Marburg}
%\usetheme{Montpellier}
%\usetheme{PaloAlto}
%\usetheme{Pittsburgh}
%\usetheme{Rochester}
%\usetheme{Singapore}
%\usetheme{Szeged}
%\usetheme{Warsaw}

% As well as themes, the Beamer class has a number of color themes
% for any slide theme. Uncomment each of these in turn to see how it
% changes the colors of your current slide theme.

%\usecolortheme{albatross}
%\usecolortheme{beaver}
%\usecolortheme{beetle}
%\usecolortheme{crane}
%\usecolortheme{dolphin}
%\usecolortheme{dove}
%\usecolortheme{fly}
%\usecolortheme{lily}
%\usecolortheme{orchid}
%\usecolortheme{rose}
%\usecolortheme{seagull}
%\usecolortheme{seahorse}
%\usecolortheme{whale}
%\usecolortheme{wolverine}

%\setbeamertemplate{footline} % To remove the footer line in all slides uncomment this line
%\setbeamertemplate{footline}[page number] % To replace the footer line in all slides with a simple slide count uncomment this line

%\setbeamertemplate{navigation symbols}{} % To remove the navigation symbols from the bottom of all slides uncomment this line
}

\usepackage{graphicx} % Allows including images
\usepackage{booktabs} % Allows the use of \toprule, \midrule and \bottomrule in tables

%\usepackage{multimedia}
\usepackage{animate,media9,movie15}
\renewcommand{\footnotesize}{\fontsize{6pt}{11pt}}


%----------------------------------------------------------------------------------------
%	TITLE PAGE
%----------------------------------------------------------------------------------------

\title[Brain dataset]{Final Report: Brain Dataset} % The short title appears at the bottom of every slide, the full title is only on the title page

\author{Roman Podolski, Philipp Bergmann, Dominik Irimi, Manuel Nickel, Christoph Dehner} % Your name
\institute[TUM] % Your institution as it will appear on the bottom of every slide, may be shorthand to save space
{
Technische Universit\"at M\"unchen \\ % Your institution for the title page
\medskip
\textit{roman.podolski@tum.de, philipp.bergmann@tum.de, dominik.irimi@tum.de, manuel.nickel@tum.de, dehner@in.tum.de} % Your email address
}
\date{\today} % Date, can be changed to a custom date

\begin{document}

\begin{frame}
\titlepage % Print the title page as the first slide
\end{frame}

\begin{frame}
\frametitle{Overview} % Table of contents slide, comment this block out to remove it
\tableofcontents % Throughout your presentation, if you choose to use \section{} and \subsection{} commands, these will automatically be printed on this slide as an overview of your presentation
\end{frame}

%----------------------------------------------------------------------------------------
%	PRESENTATION SLIDES
%----------------------------------------------------------------------------------------

%------------------------------------------------
\section{Dataset} % Sections can be created in order to organize your presentation into discrete blocks, all sections and subsections are automatically printed in the table of contents as an overview of the talk
%------------------------------------------------

%\subsection{Subsection Example} % A subsection can be created just before a set of slides with a common theme to further break down your presentation into chunks

\begin{frame}
\frametitle{Dataset}
Experiment recording human grasp and lift tasks\footnote{Data source: Luciw, M. D., Jarocka, E. \& Edin, B. B. FigShare http://dx.doi.org/10.6084/m9.figshare.988376 (2014).}
\begin{itemize}
	\item 12 participants, 9 series recorded each
	\item EEG: 32 electrodes recorded at 5kHz
	\item EMG: 5 signals at 4kHz
	\item kinetic: 36 signals at 500Hz
	\item objects to grasp with different surface friction/weights (165g - 660g)
	\item preprocessing: trials provided in windowed format (event timing relative to window)
\end{itemize}
\end{frame}


\begin{frame}
\frametitle{Experiment}

Single trial procedure
\begin{itemize}
	\item start command signaled visually by LED
	\item participant moves hand to object
    \item grasp object
    \item move object to target position
    \item hold position
    \item LED signal, to move object back to initial position
    \item hand release object
    \item move hand back to resting position
\end{itemize}
\end{frame}

\begin{frame}
    \frametitle{Experiment}
    \begin{figure}
        \includemovie[autostart,continue,controls,poster,text={\includegraphics[scale=0.45]    {images/experiment.png}}]{6cm}{6cm}{images/sdata201447-s4.avi}
    \end{figure}
\end{frame}


%------------------------------------------------
\section{Methodology}

\begin{frame}
\frametitle{Methodology}
t-SNE
\begin{itemize}
    \item expectation
    \item result
\end{itemize}
\end{frame}


\begin{frame}
\frametitle{Methodology}
General developement
\begin{itemize}
    \item Theano (python)
    \item climin library
    \item breze library
\end{itemize}

Data preparation
\begin{itemize}
    \item Input normalization to -1/1 range (tanh activation optimization)
    \item imagine one lifting trial as a single learning sample
    \begin{itemize}
        \item recordings have different length, therefore equalize it!
        \item zero padding $\rightarrow$ learning in danger of being misguided
        \item tail cut $\rightarrow$ targets fall of, fails to learn sometimes
    \end{itemize}
    \item separation into sets of 300 data point records length \textbf{improves learning.... Why did Smagt tell us to do that??)}
   	\item Subsampling (10Hz) of EMG data
   	\item Data set split: train/valid/test $\rightarrow$ 0.8/0.1/0.1
\end{itemize}
\end{frame}

\begin{frame}
\frametitle{Methodology}
Recurrent Neural Network
\begin{itemize}
    \item assuming predictability in human planning $\rightarrow$ history matters
    \item Network design:
    \begin{itemize}
        \item 100 neurons
        \item 1 hidden layer
        \item tanh activation function
        \item 50 samples per batch
        \item optimizer: Adadelta
    \end{itemize}
    \item parameter weight initialization by uniform normal distribution
    \item spectral radius
    \item Important weights: some samples are more important than others
%    \begin{itemize}
%        \item some samples are more important to learn than others
%        \item skip first ~150 samples till transient oscillation
%    \end{itemize}
    \item Bernoulli cross entropy loss% at output layer
\end{itemize}
\end{frame}

\begin{frame}
\frametitle{Methodology}
Learning Targets
\begin{itemize}
  	\item one dim multi class vector vs. mult dim one-hot-encoding

    \item Selection of available events
    \begin{enumerate}
        \item LEDOn/LEDOff: participant signal
        \item tHandStart/tHandStop: hand moving
        \item trial\_DurReach: time needed to move hand to object
        \item tLiftOff: lift object
        %\item LEDOff: end signal, return object
        %\item tHandStop: hand stops moving
        %\item tBothRelease: finger release object
    \end{enumerate}
 
    \item Targets - defined over (multiple) intervals in between events
    \begin{enumerate}
    \item move hand to target
        \item lift object
        \item hold object phase
        \item replace object
        %\item move hand to start
%        \item touch object phase
    \end{enumerate}

    \item Data shape
    \begin{itemize}
        \item Input: [time slice, features, sensors] $\rightarrow$ [300 x 2428 x 5/32]
        \item Target: [time slice, features, targets] $\rightarrow$ [300 x 1320 x 1]
%        \item format given as breze requirement \textbf{(Does anyone in the audience care?)}
    \end{itemize}
\end{itemize}

%LSTM
%\begin{itemize}
%    \item usage of breze library implementation \footnote{https://github.com/breze-no-salt/breze v0.1 (2016)} 
%    \item finally not used because satisfying results achivable by RNN
%\end{itemize}

\end{frame}

%------------------------------------------------
\section{Results}



		
		
		
\begin{frame}
	\frametitle{Results (1): t-SNE}
	\begin{itemize}
		\item Figure of t-SNE of EEG data $\rightarrow$ Seperability of trials
		\item Possible to separate with standard NN
		\item Figure of t-SNE of EMG data $\rightarrow$ As expected
	\end{itemize}
\end{frame}

\begin{frame}
	\frametitle{Results (2): RNN}
	\begin{itemize}
		\item Overview of the targets
		\item Hand move to target works good.
		\item Touch phase target also quite ok.
		\item hand move back target also (partially) sucessful
		\item Comparison: Training with data of one person vs. data of more participants
		\item etc.
	\end{itemize}
\end{frame}

%------------------------------------------------
\section{Pitfalls}

\begin{frame}
	\frametitle{Pitfalls}
	\begin{itemize}
		\item Prediction do not fit to the target borders exactly
		\item No working early stopping criterion (so far)
		\item Targets within the lift phase cannot be predicted properly
		\item etc.
	\end{itemize}
\end{frame}

%------------------------------------------------

\begin{frame}
\frametitle{Blocks of Highlighted Text}
\begin{block}{Block 1}
Lorem ipsum dolor sit amet, consectetur adipiscing elit. Integer lectus nisl, ultricies in feugiat rutrum, porttitor sit amet augue. Aliquam ut tortor mauris. Sed volutpat ante purus, quis accumsan dolor.
\end{block}

\begin{block}{Block 2}
Pellentesque sed tellus purus. Class aptent taciti sociosqu ad litora torquent per conubia nostra, per inceptos himenaeos. Vestibulum quis magna at risus dictum tempor eu vitae velit.
\end{block}

\begin{block}{Block 3}
Suspendisse tincidunt sagittis gravida. Curabitur condimentum, enim sed venenatis rutrum, ipsum neque consectetur orci, sed blandit justo nisi ac lacus.
\end{block}
\end{frame}

%------------------------------------------------

\begin{frame}
\frametitle{Multiple Columns}
\begin{columns}[c] % The "c" option specifies centered vertical alignment while the "t" option is used for top vertical alignment

\column{.45\textwidth} % Left column and width
\textbf{Heading}
\begin{enumerate}
\item Statement
\item Explanation
\item Example
\end{enumerate}

\column{.5\textwidth} % Right column and width
Lorem ipsum dolor sit amet, consectetur adipiscing elit. Integer lectus nisl, ultricies in feugiat rutrum, porttitor sit amet augue. Aliquam ut tortor mauris. Sed volutpat ante purus, quis accumsan dolor.

\end{columns}
\end{frame}

\begin{frame}
\frametitle{Table}
\begin{table}
\begin{tabular}{l l l}
\toprule
\textbf{Treatments} & \textbf{Response 1} & \textbf{Response 2}\\
\midrule
Treatment 1 & 0.0003262 & 0.562 \\
Treatment 2 & 0.0015681 & 0.910 \\
Treatment 3 & 0.0009271 & 0.296 \\
\bottomrule
\end{tabular}
\caption{Table caption}
\end{table}
\end{frame}

%------------------------------------------------

\begin{frame}
\frametitle{Theorem}
\begin{theorem}[Mass--energy equivalence]
$E = mc^2$
\end{theorem}
\end{frame}

%------------------------------------------------

\begin{frame}[fragile] % Need to use the fragile option when verbatim is used in the slide
\frametitle{Verbatim}
\begin{example}[Theorem Slide Code]
\begin{verbatim}
\begin{frame}
\frametitle{Theorem}
\begin{theorem}[Mass--energy equivalence]
$E = mc^2$
\end{theorem}
\end{frame}\end{verbatim}
\end{example}
\end{frame}

%------------------------------------------------

\begin{frame}
\frametitle{Figure}
Uncomment the code on this slide to include your own image from the same directory as the template .TeX file.
%\begin{figure}
%\includegraphics[width=0.8\linewidth]{test}
%\end{figure}
\end{frame}

%------------------------------------------------

\begin{frame}[fragile] % Need to use the fragile option when verbatim is used in the slide
\frametitle{Citation}
An example of the \verb|\cite| command to cite within the presentation:\\~

This statement requires citation \cite{p1}.
\end{frame}

%------------------------------------------------

\begin{frame}
\frametitle{References}
\footnotesize{
\begin{thebibliography}{99} % Beamer does not support BibTeX so references must be inserted manually as below
\bibitem[Smith, 2012]{p1} John Smith (2012)
\newblock Title of the publication
\newblock \emph{Journal Name} 12(3), 45 -- 678.
\end{thebibliography}
}
\end{frame}

%------------------------------------------------

\begin{frame}
\Huge{\centerline{The End}}
\end{frame}

%----------------------------------------------------------------------------------------

\end{document} 
